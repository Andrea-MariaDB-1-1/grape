%%%%%%%%%%%%%%%%%%%%%%%%%%%%%%%%%%%%%%%%%%%%%%%%%%%%%%%%%%%%%%%%%%%%%%%%%%%%
%
%A  grape.tex               GRAPE documentation              Leonard Soicher
%
%
%
\def\GRAPE{\sf GRAPE}
\def\nauty{\it nauty}
\def\Aut{{\rm Aut}\,} 

\Chapter{Grape}

This manual describes the {\GRAPE} (Version~4.6) package for computing
with graphs and groups.

{\GRAPE} is primarily designed for the construction and analysis of
finite graphs related to groups, designs, and geometries. Special
emphasis is placed on the determination of regularity properties and
subgraph structure. The {\GRAPE} philosophy is that a graph <gamma>
always comes together with a known subgroup <G> of the automorphism
group of <gamma>, and that <G> is used to reduce the storage and
CPU-time requirements for calculations with <gamma> (see
\cite{Soi93} and \cite{Soi04}).  Of course <G> may be the trivial group,
and in this case {\GRAPE} algorithms may perform more slowly than strictly
combinatorial algorithms (although this degradation in performance is
hopefully never more than a fixed constant factor).

Most {\GRAPE} functions are written entirely in the {\GAP} language.
However, the {\GRAPE} functions `AutomorphismGroup', `AutGroupGraph',
`IsIsomorphicGraph', `GraphIsomorphismClassRepresentatives',
`GraphIsomorphism'  and `PartialLinearSpaces' make direct or indirect use
of B.D.~McKay{\pif}s {\nauty} (Version~2.2 final) package \cite{Nau90},
via a {\GRAPE} interface.  These functions can only be used on a fully
installed version of {\GRAPE}. Installation of {\GRAPE} is described
in its `README' file and in its manual section "Installing the GRAPE
Package".

Except for the {\nauty} package included with {\GRAPE}, the function
`SmallestImageSet' by Steve Linton, and the new {\nauty} interface
by Alexander Hulpke, the {\GRAPE} package was designed and written
by Leonard H. Soicher, School of Mathematical Sciences, Queen Mary,
University of London.

If you use {\GRAPE} to solve a problem then please send a short email
about it to \Mailto{L.H.Soicher@qmul.ac.uk}, and reference the {\GRAPE} 
package as follows:

L.H. Soicher, The {GRAPE} package for {GAP}, Version~4.6, 2012,
\URL{http://www.maths.qmul.ac.uk/~leonard/grape/}.

If your work made use of a function depending on the {\nauty} package
then you should also reference {\nauty} \cite{Nau90}.

The development of {\GRAPE} was partially supported by a European Union
HCM grant in ``Computational Group Theory''.

%%%%%%%%%%%%%%%%%%%%%%%%%%%%%%%%%%%%%%%%%%%%%%%%%%%%%%%%%%%%%%%%%%%%%%%%%
\Section{Installing the GRAPE Package}

The {\GAP}~4.5 distribution includes the {\GRAPE} package, which now
includes a 32-bit nauty/dreadnaut binary for Windows (XP and later
versions).  Thus, {\GRAPE} normally requires no further installation
for Windows users of {\GAP}~4.5.

You do not need to download and unpack an archive for {\GRAPE}
unless you want to install the package separately from your main
{\GAP} installation or are installing an upgrade of {\GRAPE} to an
existing installation of {\GAP} (see the main {\GAP} reference section
"Reference:Installing a GAP Package").  If you do need to download
{\GRAPE}, you can find archive files for the package in various formats
at \URL{http://www.gap-system.org/Packages/grape.html}, and then your
archive file of choice should be downloaded and unpacked in the `pkg'
subdirectory of an appropriate {\GAP} root directory (see the main {\GAP}
reference section "Reference:GAP Root Directories").

Unless you are running {\GRAPE} under Windows (XP or later), you will
normally need to perform compilation of B.D.~McKay's nauty/dreadnaut
programs included with {\GRAPE}, and in a Unix environment, you should
proceed as follows.  After installing {\GAP}, go to the {\GRAPE}
home directory (usually the directory `pkg/grape' of the {\GAP} home
directory), and run `./configure <path>', where <path> is the path of the
{\GAP} home directory.  So for example, if the {\GRAPE} home directory
is the `pkg/grape' directory of the {\GAP} home directory, run
\begintt 
./configure ../..  
\endtt
This will create a `Makefile'. Now run
\begintt
make
\endtt
to create the nauty/dreadnaut binary and to put it in the appropriate place.

You should test {\GRAPE} and its interface to {\nauty}.  Start up
{\GAP} and at the prompt type 
\begintt 
LoadPackage( "grape" ); 
\endtt
On-line documentation for {\GRAPE} should be available by typing 
\begintt
?GRAPE 
\endtt 
The command 
\begintt 
IsIsomorphicGraph( JohnsonGraph(7,3), JohnsonGraph(7,4) ); 
\endtt 
should return `true', and 
\begintt 
Size( AutGroupGraph( JohnsonGraph(4,2) ) ); 
\endtt 
should be `48'.

Both dvi and pdf versions of the {\GRAPE} manual are available
(as `manual.dvi' and `manual.pdf' respectively) in the `doc' directory
of the home directory of {\GRAPE}.

If you install {\GRAPE}, then please tell \Mailto{L.H.Soicher@qmul.ac.uk},
where you should also send any comments or bug reports.

%%%%%%%%%%%%%%%%%%%%%%%%%%%%%%%%%%%%%%%%%%%%%%%%%%%%%%%%%%%%%%%%%%%%%%%%%
\Section{Loading GRAPE}

Before using {\GRAPE} you must load the package within {\GAP} by calling 
the statement

\begintt
gap> LoadPackage("grape");
true
\endtt

%%%%%%%%%%%%%%%%%%%%%%%%%%%%%%%%%%%%%%%%%%%%%%%%%%%%%%%%%%%%%%%%%%%%%%%%%
\Section{The structure of a graph in GRAPE}
 
In general {\GRAPE} deals with finite directed graphs which may have
loops but have no multiple edges. However, many {\GRAPE} functions only
work for *simple* graphs (i.e. no loops, and whenever $[x,y]$ is an
edge then so is $[y,x]$), but these functions will check if an input
graph is simple.

In {\GRAPE}, a graph <gamma> is stored as a record, with mandatory
components `isGraph', `order', `group', `schreierVector',
`representatives', and `adjacencies'. Usually, the user need not be
aware of this record structure, and is strongly advised only to use
{\GRAPE} functions to construct and modify graphs.

The `order' component contains the number of vertices of <gamma>. The
vertices of <gamma> are always 1,2,...,`<gamma>.order', but they may also
be given *names*, either by a user (using `AssignVertexNames') or by a
function constructing a graph (e.g. `InducedSubgraph', `BipartiteDouble',
`QuotientGraph'). The `names' component, if present, records these
names, with `<gamma>.names[<i>]' the name of vertex <i>.  If the `names'
component is not present (the user may, for example, choose to unbind
it), then the names are taken to be 1,2,...,`<gamma>.order'. The `group'
component records the {\GAP} permutation group associated with <gamma>
(this group must be a subgroup of the automorphism group of <gamma>). The
`representatives' component records a set of orbit representatives
for the action of `<gamma>.group' on the vertices of <gamma>, with
`<gamma>.adjacencies[<i>]' being the set of vertices adjacent to
`<gamma>.representatives[<i>]'. The `group' and `schreierVector'
components are used to compute the adjacency-set of an arbitrary vertex
of <gamma> (this is done by the function `Adjacency').

The only mandatory component which may change once a graph is initially
constructed is `adjacencies' (when an edge-orbit of `<gamma>.group' is
added to, or removed from, <gamma>). A graph record may also have some
of the optional components `isSimple', `autGroup', and
`canonicalLabelling', which record information about that graph.

%%%%%%%%%%%%%%%%%%%%%%%%%%%%%%%%%%%%%%%%%%%%%%%%%%%%%%%%%%%%%%%%%%%%%%%%
\Section{Examples of the use of GRAPE}

We give here a simple example to illustrate the use of {\GRAPE}. All
functions used are described in detail in this manual. More
sophisticated examples of the use of {\GRAPE} can be found in
chapter "Partial Linear Spaces", and also in the references \cite{Cam99},
\cite{CSS99}, \cite{HL99} and \cite{Soi06}.

In the example here, we construct the Petersen graph $P$, and its edge
graph (also called line graph) $EP$. We compute the global parameters
of $EP$, and so verify that $EP$ is distance-regular (see \cite{BCN89}).

\beginexample
gap> LoadPackage("grape");
true
gap> P := Graph( SymmetricGroup(5), [[1,2]], OnSets,
>             function(x,y) return Intersection(x,y)=[]; end );
rec( isGraph := true, order := 10, 
  group := Group([ ( 1, 2, 3, 5, 7)( 4, 6, 8, 9,10), ( 2, 4)( 6, 9)( 7,10) ]),
  schreierVector := [ -1, 1, 1, 2, 1, 1, 1, 1, 2, 2 ], 
  adjacencies := [ [ 3, 5, 8 ] ], representatives := [ 1 ], 
  names := [ [ 1, 2 ], [ 2, 3 ], [ 3, 4 ], [ 1, 3 ], [ 4, 5 ], [ 2, 4 ], 
      [ 1, 5 ], [ 3, 5 ], [ 1, 4 ], [ 2, 5 ] ] )
gap> Diameter(P);
2
gap> Girth(P);
5
gap> EP := EdgeGraph(P);
rec( isGraph := true, order := 15, 
  group := Group([ ( 1, 4, 7, 2, 5)( 3, 6, 8, 9,12)(10,13,14,15,11), 
      ( 4, 9)( 5,11)( 6,10)( 7, 8)(12,15)(13,14) ]), 
  schreierVector := [ -1, 1, 1, 1, 1, 1, 1, 2, 2, 1, 2, 1, 1, 1, 2 ], 
  adjacencies := [ [ 2, 3, 7, 8 ] ], representatives := [ 1 ], 
  isSimple := true, 
  names := [ [ [ 1, 2 ], [ 3, 4 ] ], [ [ 1, 2 ], [ 4, 5 ] ], 
      [ [ 1, 2 ], [ 3, 5 ] ], [ [ 2, 3 ], [ 4, 5 ] ], [ [ 2, 3 ], [ 1, 5 ] ], 
      [ [ 2, 3 ], [ 1, 4 ] ], [ [ 3, 4 ], [ 1, 5 ] ], [ [ 3, 4 ], [ 2, 5 ] ], 
      [ [ 1, 3 ], [ 4, 5 ] ], [ [ 1, 3 ], [ 2, 4 ] ], [ [ 1, 3 ], [ 2, 5 ] ], 
      [ [ 2, 4 ], [ 1, 5 ] ], [ [ 2, 4 ], [ 3, 5 ] ], [ [ 3, 5 ], [ 1, 4 ] ], 
      [ [ 1, 4 ], [ 2, 5 ] ] ] )
gap> GlobalParameters(EP);
[ [ 0, 0, 4 ], [ 1, 1, 2 ], [ 1, 2, 1 ], [ 4, 0, 0 ] ]
\endexample
